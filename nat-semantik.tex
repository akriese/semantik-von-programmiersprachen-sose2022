%%%%%%%%%%%%%%%%%%%%%%%%%%%%%%%%%%%%%%%%%%%%%%%%%%%%%%%%%%%%%%%%%%%%%
\newpage
\hfill 20.05.
%%%%%%%%%%%%%%%%%%%%%%%%%%%%%%%%%%%%%%%%%%%%%%%%%%%%%%%%%%%%%%%%%%%%%
\subsection{Natürliche operationelle Semantik (``big step'')}
% \subsection{Definition von \texorpdfstring{$\mathcal{S}\lsem \cdot \rsem$}{S[.]}}

$\Asem{\cdot}$ und $\Bsem{\cdot}$ \emph{benutzen} den Zustand $\sigma$ während die Semantik von Anweisungen den Zustand \emph{verändern} soll.

Idee: Definiere $\Ssem{\cdot}$ mithilfe einer Zustandsüberführungerelation.

\begin{definition}
    Sei $s \in \Stm$ eine Anweisungsfolge und seien $\sigma, \sigma' \in \State$ Zustände. Die Zustandsüberführungsrelation
    \[
    \strans{S}{\sigma}{\sigma'}
    \]
    spezifiziert die Beziehung zwischen Startzustand $\sigma$ und dem Endzustand $\sigma'$ gemäß der Anweisungsfolge $S$.
\end{definition}

\begin{remark}[Bedeutung]
    Die Ausführung von $S$ auf Startzustand $\sigma$ terminiert mit Endzustand $\sigma'$.
\end{remark}

\par\medskip
\begin{notation}
    Wir definieren die Zustandsübergangsrelation mithilfe von Schlussregeln. Eine solche Schlussregel besitzt die folgende Form:
    \[
    \frac{\text{Voraussetzung}}{\text{Folgerung}} \leadsto \frac{\strans{S_1}{\sigma_1}{\sigma_1'}, \strans{S_2}{\sigma_2}{\sigma_2'}, \,\dots\, , \strans{S_n}{\sigma_n}{\sigma_n'}}{\strans{S}{\sigma}{\sigma'}}
    \]
    (ggf.\ mit zusätzlichen Bedingungen) wobei $S_1, \dots, S_n$ Bestandteile von $S$ sind.

    Gibt es keine Voraussetzung, nennen wir die Schlussregel ein \emph{Axiom}.
\end{notation}



% SCHLUSSREGELN %%%%%%%%%%%%%%%%%%%%%%%%%%%%%%%%%%%%%%%%%%%%%%%%%%%%%%%%%%%%%%%%%%%%%%%%%%%%%%%
\subsubsection{Schlussregeln für die natürliche Semantik von \texttt{while}}

Im Folgenden schreiben wir $[\cdot]_{\text{ns}}$ um anzuzeigen, dass es sich um die natürliche Semantik handelt.

\begin{enumerate}
    \item Zuweisung $\infruleNs{zuw}$ (Axiom)
    \[
    \frac{}{\strans{x \texttt{ := } a}{\sigma}{\sigma[x \mapsto \Asem{a}(\sigma)]}}
    \]

    \item Skip $\infruleNs{skip}$ (Axiom)
    \[
    \frac{}{\strans{\texttt{skip}}{\sigma}{\sigma}}
    \]

    \item Hintereinanderausführung $\infruleNs{seq}$
    \[
    \frac{\strans{S_1}{\sigma}{\sigma'}, \strans{S_2}{\sigma'}{\sigma''}}{\strans{S_1 \texttt{;} S_2}{\sigma}{\sigma''}}
    \]
    Dabei können $S_1$ und $S_2$ zusammengesetzte Anweisungen sein.

    \item Verzweigung $\infruleNs[\true]{if}$
    \[
    \frac{\strans{S_1}{\sigma}{\sigma'}}{\strans{\texttt{if } b \texttt{ then } S_1 \texttt{ else } S_2}{\sigma}{\sigma'}}
    \]
    falls $\Bsem{b}(\sigma) = \true$. Dabei muss $S_2$ nicht terminieren.

    \item Verzweigung $\infruleNs[\false]{if}$
    \[
    \frac{\langle S_2, \sigma \rangle \to \sigma'}{\langle \texttt{if } b \texttt{ then } S_1 \texttt{ else } S_2, \sigma \rangle \to \sigma'}
    \]
    falls $\Bsem{b}(\sigma) = \false$. Dabei muss $S_1$ nicht terminieren.

    \item Schleife $\infruleNs[\true]{while}$
    \[
    \frac{\langle S, \sigma \rangle \to \sigma', \langle \texttt{while } b \texttt{ do } S, \sigma' \rangle \to \sigma''}{\langle \texttt{while } b \texttt{ do } S, \sigma \rangle \to \sigma''}
    \]
    falls $\Bsem{b}(\sigma) = \true$. Im Allgemeinen kann es sein, dass die Schleife nicht terminiert. Deshalb müssen wir diesen Fall in der Definition der semantischen Funktion beachten. Das bedeutet, diese Relation ist nicht total.

    \item Schleife $\infruleNs[\false]{while}$
    \[
    \frac{\text{ }}{\langle \texttt{while } b \texttt{ do } S, \sigma \rangle \to \sigma}
    \]
    falls $\Bsem{b}(\sigma) = \false$.
\end{enumerate}

\par\bigskip
\par\bigskip
Wie ist die Zustandsüberführungsfunktion überhaupt definiert?

\begin{definition}
    Sei $S \in \Stm$ ein Programm und seien $\sigma, \sigma' \in \State$. Dann gilt
    \[
    \langle S, \sigma \rangle \to \sigma'
    \]
    gdw.\ ein \emph{endlicher Ableitungsbaum} dafür existiert.

    Der Ableitungsbaum entsteht durch wiederholte Anwendung der Schlussregeln. Die Wurzel ist $\langle S, \sigma \rangle \to \sigma'$, die Blätter sind Axiome, die Knoten entsprechen der korrekten Anwendung der Schlussregeln.
\end{definition}

\begin{example}
    Sei $\sigma \in \State$ mit $\sigma(x) = 1$ und $\sigma(y) = 5$.

    Behauptung: $\langle (z := z; x := y); y := z, \sigma \rangle \to \sigma[z \mapsto 1][x \mapsto 5][y \mapsto 1]$

    Nun müssen wir den endlichen Ableitungsbaum erzeugen.


    \begin{enumerate}
        \item $\infruleNs{seq}$
            \[
            \frac{\langle z := x; x ;= y, \sigma \rangle \to \sigma', \langle y := z, \sigma' \rangle \to \sigma[z \mapsto 1][x \mapsto 5][y \mapsto 1]}{\langle \underbrace{(z := z; x := y)}_{s_1}; \underbrace{y := z}_{S_2}, \sigma \rangle \to \sigma[z \mapsto 1][x \mapsto 5][y \mapsto 1]}
            \]

            Welches $\sigma'$ brauchen wir? $\leadsto \sigma[z \mapsto 1][x \mapsto 5]$. Somit erhalten wir
            \[
            \frac{\langle z := x; x := y, \sigma \rangle \to \sigma[z \mapsto 1][x \mapsto 5], \langle y := z, \sigma[z \mapsto 1][x \mapsto 5] \rangle \to \sigma[z \mapsto 1][x \mapsto 5][y \mapsto 1]}{\langle \underbrace{(z := z; x := y)}_{S_1}; \underbrace{y := z}_{S_2}, \sigma \rangle \to \sigma[z \mapsto 1][x \mapsto 5][y \mapsto 1]}
            \]

        \item $\infruleNs{zuw}$ für den rechten Teil
            \[
            \frac{\Asem{z}(\sigma[z \mapsto 1][x \mapsto 5]) \overset{?}{=} 1}{\langle y := z, \sigma[z \mapsto 1][x \mapsto 5] \rangle \to \sigma[z \mapsto 1][x \mapsto 5][y \mapsto 1]}
            \]

            \[
            \Asem{z}(\sigma[z \mapsto 1][x \mapsto 5]) \overset{Def}{=} \sigma[z \mapsto 1][x \mapsto 5](z) \overset{Def}{=} 1
            \]

        \item $\infruleNs{seq}$
            \[
            \frac{\langle z := x, \sigma \rangle \to \sigma[z \mapsto 1], \langle x := y, \sigma[z \mapsto 1] \rangle \to \sigma[z \mapsto 1][x \mapsto 5]}{\langle z := x; x ;= y, \sigma \rangle \to \sigma[z \mapsto 1][x \mapsto 5]}
            \]

            Ab hier analog zum rechten Teil (mit zwei Mal $\infruleNs{zuw}$).
    \end{enumerate}
\end{example}

\par\bigskip
\begin{remark}[Rechtseindeutigkeit / Determiniertheit]
    Es ist noch nicht klar, dass ``$\to$'' \emph{rechtseindeutig} ist. D.\,h. möglicherweise existiert ein Programm $S$, ein Startzustand $\sigma$ und Zustände $\sigma_1 \neq \sigma_2 \in \State$, sodass sowohl
    \[
    \langle S, \sigma \rangle \to \sigma_1 \text{ als auch } \langle S, \sigma \rangle \to \sigma_2
    \]
    gilt. Daher muss die Rechtseindeutigkeit bzw. \emph{Determiniertheit} bewiesen werden!
\end{remark}

\par\medskip
\begin{remark}[Ableitungsbaum]
    Der Ableitungsbaum ist statisch. D.\,h. man kann nicht erkennen, in welcher Reihenfolge die Schlussregeln angewendet werden.
\end{remark}



%%%%%%%%%%%%%%%%%%%%%%%%%%%%%%%%%%%%%%%%%%%%%%%%%%%%%%%%%%%%%%%%%%%%%
\newpage
\hfill 27.05.
%%%%%%%%%%%%%%%%%%%%%%%%%%%%%%%%%%%%%%%%%%%%%%%%%%%%%%%%%%%%%%%%%%%%%
\begin{definition}
    Sei $S$ ein Programm und $\sigma$ ein Startzustand. Das Programm $S$ \emph{terminiert} bei Startzustand $\sigma$ falls ein $\sigma'$ existiert, sodass $\strans{S}{\sigma}{\sigma'}$ gilt.

    ? Analog \emph{terminiert $S$ nicht} bei Startzustand $\sigma$.

    Das Programm $S$ \emph{terminiert immer}, falls $S$ für jeden Startzustand $\sigma$ terminiert. $S$ \emph{terminiert nie}, falls $S$ für keinen Startzustand $\sigma$ terminiert.
\end{definition}

\begin{definition}
    Seien $S_1, S_2$ zwei Programme. $S_1$ und $S_2$ heißen \emph{semantisch äquivalent}, falls für alle Zustände $\sigma, \sigma'$ gilt
    \[
    \strans{S_1}{\sigma}{\sigma'} \Leftrightarrow \strans{S_2}{\sigma}{\sigma'}
    \]
\end{definition}

\begin{example}
    Die Programme
    \[
    S_1 = \texttt{while b do S}
    \]
    und
    \[
    S_2 = \texttt{if b then (S; while b do S) else skip}
    \]
    sind semantisch äquivalent.
\end{example}
\begin{proof}
    Seien $\sigma, \sigma'$ Zustände. Z.\,z.: $\strans{S_1}{\sigma}{\sigma'} \Leftrightarrow \strans{S_2}{\sigma}{\sigma'}$.

    \begin{enumerate}
        \item ``$\Rightarrow$''

            Angenommen, es gilt $\strans{S_1}{\sigma}{\sigma'}$. Also existiert nach Definition ein endlicher Ableitungsbaum für $\strans{S_1}{\sigma}{\sigma'}$.
            \begin{enumerate}
                \item $\Bsem{b}(\sigma) = \true$

                    Dann hat der Ableitungsbaum $T_1$ die Form
                    \begin{align*}
                        \infruleNs[\true]{while}\;
                        \cfrac{
                            \cfrac{T_a}{\strans{S}{\sigma}{\sigma''}}
                            \;,\;
                            \cfrac{T_b}{\strans{\texttt{while b do S}}{\sigma''}{\sigma'}}
                        }{
                            \strans{\texttt{while b do S}}{\sigma}{\sigma'}
                        }
                    \end{align*}

                    Z.\,z.:\ es existiert ein endlicher Ableitungsbaum für $\strans{S_2}{\sigma}{\sigma'}$
                    \begin{align*}
                        \infruleNs[\true]{if}\;
                        \cfrac{
                            \infruleNs{seq}\;
                            \cfrac{
                                \cfrac{T_a}{\strans{S}{\sigma}{\sigma''}}
                                ,
                                \cfrac{T_b}{\strans{\texttt{while b do S}}{\sigma''}{\sigma'}}
                            }{
                                \strans{\texttt{S; while b do S}}{\sigma}{\sigma'}
                            }
                        }{
                            \strans{\texttt{if b then (S; while b do S) else skip}}{\sigma}{\sigma'}
                        }
                        \quad
                        \Bsem{b}(\sigma) = \true
                    \end{align*}

                \item $\Bsem{b}(\sigma) = \false$

                    Dann hat $T_1$ die Form
                    \begin{align*}
                        \infruleNs[\false]{while}\;
                        \cfrac{}{
                            \strans{\texttt{while b do S}}{\sigma}{\sigma}
                        }
                        \quad
                        \Bsem{b}(\sigma) = \false
                    \end{align*}
                    Dieses Axiom ist wahr, gdw. $\sigma = \sigma'$.

                    Z.\,z.: Es existiert ein Ableitungsbaum für $\strans{S_2}{\sigma}{\sigma}$.
                    \begin{align*}
                        \infruleNs[\false]{if}\;
                        \cfrac{
                            \infruleNs{skip}\;
                            \cfrac{}{\strans{\texttt{skip}}{\sigma}{\sigma}}
                        }{
                            \strans{\texttt{if b then (S; while b do S) else skip}}{\sigma}{\sigma}
                        }
                        \quad
                        \Bsem{b}(\sigma) = \false
                    \end{align*}
            \end{enumerate}

        \item ``$\Leftarrow$''

            Analog.
    \end{enumerate}
\end{proof}

\begin{lemma}
    Die natürliche Semantik der \texttt{while}-Sprache ist \emph{determiniert}, \dh{} für alle Anweisungen $S$ und Zustände $\sigma, \sigma_1, \sigma_2$ gilt:

    Wenn $\strans{S}{\sigma}{\sigma_1}$ und $\strans{S}{\sigma}{\sigma_2}$, dann $\sigma_1 = \sigma_2$.
\end{lemma}
\begin{proof}
    Durch strukturelle Induktion nach Tiefe des Ableitungsbaums. (Übung)
\end{proof}


\begin{definition}
    Die semantische Funktion $\mathcal{S}_{\text{ns}}: \State \to (\State \to \State)$ ist definiert als
    \begin{align*}
        \mathcal{S}_{\text{ns}}\lsem S \rsem(\sigma) = \begin{cases}
            \sigma' & \text{falls } \exists\; \sigma': \strans{S}{\sigma}{\sigma'} \\
            \bot & \text{sonst}
        \end{cases}
    \end{align*}
\end{definition}