\subsection{Strukturelle operationelle Semantik (``small step'')}

Hier geht es um die genaue Reihenfolge der Schritte bei der Ausführung. Das ist bespielsweise nützlich bei der parallelen Ausführungen eines Programms.

Wir definieren wieder eine Zustandsüberführungsrelation ``$\Rightarrow$''. Sie hat die Form
\begin{align*}
    \langle S, \sigma \rangle & \Rightarrow \langle S', \sigma' \rangle \tag{*} \\
    \text{oder} \\
    \langle S, \sigma \rangle & \Rightarrow \sigma' \tag{**}
\end{align*}

Interpretation:
\begin{enumerate}
    \item[(*)] Ausführung ist noch nicht vorbei, sondern erreicht in \emph{einem Schritt} die \emph{Zwischenkonfiguration} $\langle s', \sigma' \rangle$.
    \item[(**)] Ausführung ist nach einem Schritt vorbei und erreicht den Endzustand $\sigma'$.
\end{enumerate}



Wir definieren $\Rightarrow$ durch folgende Schlussregeln. Im Folgenden schreiben wir $[\cdot]_{\text{sos}}$ um anzuzeigen, dass es sich um die strukturelle operationelle Semantik handelt.


\begin{enumerate}
    \item $\infruleSos{zuw}$
    \[
    \langle x \texttt{ := } a, \sigma \rangle \Rightarrow \sigma[x \mapsto \Asem{a}(\sigma)]
    \]

    \item $\infruleSos{skip}$
    \[
    \langle \texttt{skip}, \sigma \rangle \Rightarrow \sigma
    \]

    \item $\infruleSos[1]{seq}$
    \begin{align*}
        \frac{
            \langle S_1, \sigma \rangle \Rightarrow \langle S_1', \sigma' \rangle
        }{
            \langle S_1\texttt{;} S_2, \sigma \rangle \Rightarrow \langle S_1'\texttt{;} S_2, \sigma' \rangle
        }
    \end{align*}

    \item $\infruleSos[2]{seq}$
    \begin{align*}
        \frac{
            \langle S_1, \sigma \rangle \Rightarrow \sigma'
        }{
            \langle S_1\texttt{;} S_2, \sigma \rangle \Rightarrow \langle S_2, \sigma' \rangle
        }
    \end{align*}


    \item $\infruleSos[\true]{if}$
    \begin{align*}
        \langle \texttt{if } b \texttt{ then } S_1 \texttt{ else } S_2, \sigma \rangle \Rightarrow \langle S_1, \sigma \rangle
    \end{align*}
    falls $\Bsem{b}(\sigma) = \true$

    \item $\infruleSos[\false]{if}$
    \begin{align*}
        \langle \texttt{if } b \texttt{ then } S_1 \texttt{ else } S_2, \sigma \rangle \Rightarrow \langle S_2, \sigma \rangle
    \end{align*}
    falls $\Bsem{b}(\sigma) = \false$

    \item $\infruleSos{while}$
    \[
    \langle \texttt{while } b \texttt{ do } S, \sigma \rangle \Rightarrow \langle \texttt{if } b \texttt{ then } \texttt{(S; while } b \texttt{ do } S) \texttt{ else skip}, \sigma \rangle
    \]
\end{enumerate}

\begin{definition}
    Sei $S$ eine Anweisung und $\sigma$ ein Zustand.

    Eine Ableitungsfolge für $\langle S, \sigma \rangle$ ist entweder
    \begin{enumerate}
        \item eine endliche Folge $\gamma_0 \Rightarrow \gamma_1 \Rightarrow \dots \Rightarrow \gamma_k$ von Konfigurationen, sodass $\gamma_0 = \langle S, \sigma \rangle$ ist, $\gamma_i \Rightarrow \gamma_{i+1}$ für $0, \dots, k-1$ gilt und $\gamma_k$ entweder ein Zustand $\sigma'$ oder eine Konfiguration $\langle S', \sigma' \rangle$ ist, für die es mit $\Rightarrow$ nicht weiter geht \emph{(steckengebliebe Konfiguration)}.

        \item eine unendliche Folge $\gamma_0 \Rightarrow \gamma_1 \Rightarrow \dots$ von Konfigurationen mit $\gamma_0 = \langle S, \sigma \rangle$ mit $\gamma_i \Rightarrow \gamma_{i+1}$ für $i \geq 0$.
    \end{enumerate}
\end{definition}

\par\medskip
\begin{notation}
    Wir schreiben

    $\gamma_0 \Rightarrow^i \gamma'$ für ``$\gamma'$ geht aus $i$ Schritten hervor'' und

    $\gamma_0 \Rightarrow^* \gamma'$ für ``$\gamma'$ geht aus endlich vielen Schritten hervor'' (auch null).
\end{notation}



%%%%%%%%%%%%%%%%%%%%%%%%%%%%%%%%%%%%%%%%%%%%%%%%%%%%%%%%%%%%%%%%%%%%%
\newpage
\hfill 03.06.
%%%%%%%%%%%%%%%%%%%%%%%%%%%%%%%%%%%%%%%%%%%%%%%%%%%%%%%%%%%%%%%%%%%%%

\begin{example}
    Sei $\sigma$ ein Zustand mit $\sigma(x) = 5, \sigma(y) = 7$. Betrachte die Auswertung von \texttt{(z := x; x := y); y = z;} in der SOS für Startzustand $\sigma$.

    \begin{align*}
        & \langle \texttt{(z := x; x := y); y = z;}, \sigma \rangle \\
        \overset{\infruleSos[1]{seq}}{\Rightarrow} \quad & \langle \texttt{x := y; y := z}, \sigma[z \mapsto 5] \rangle \tag{i} \\
        \overset{\infruleSos[2]{seq}}{\Rightarrow} \quad & \langle \texttt{y := z}, \sigma[z \mapsto 5][y \mapsto 7] \rangle \tag{ii} \\
        \overset{\infruleSos{zuw}}{\Rightarrow} \quad & \sigma[z \mapsto 5][x \mapsto 7][y \mapsto 5]
    \end{align*}

    zu (i):
    \begin{align*}
        \infruleSos[1]{seq}\; \cfrac{
            \infruleSos[2]{seq}\; \cfrac{
                \infruleSos{zuw}\; \cfrac{}{
                    \stranssos{\texttt{z := x}}{\sigma}{\sigma[z \mapsto 5]}
                }
            }{
                \stranssos{\texttt{z := x; x := y}}{\sigma}{\langle \texttt{x := y}, \sigma[z \mapsto 5] \rangle}
            }
        }{
            \stranssos{\texttt{(z := x; x := y); y := z}}{\sigma}{\langle \texttt{x := y; y := z}, \sigma[z \mapsto 5] \rangle}
        }
    \end{align*}

    zu (ii):
    \begin{align*}
        \infruleSos[2]{seq}\; \cfrac{
            \infruleSos{zuw}\; \cfrac{}{
                \stranssos{x :=y}{\sigma[z \mapsto 5]}{\langle y :=z, \sigma[z \mapsto5][x \mapsto 7] \rangle}
            }
        }{
            \stranssos{x :=y; y :=z}{\sigma[z \mapsto 5]}{\langle y :=z, \sigma[z \mapsto5][x \mapsto 7] \rangle}
        }
    \end{align*}
\end{example}



\subsection{Eigenschaften der SOS}

\begin{lemma}
    Seiten $S_1, S_2$ Anweisungen, $\sigma, \sigma''$ Zustände und $k \in \mathbb{N}$.
    Dann gilt: Falls $\langle S_1; S_2, \sigma \rangle \Rightarrow^k \sigma''$ gilt, es existieren zwei Zahlen $k_1, k_2 \in \mathbb{N}$ mit
    \[
        \langle S_1, \sigma \rangle \Rightarrow^{k_1} \sigma'
        \;,\quad
        \langle S_2, \sigma' \rangle \Rightarrow^{k_2} \sigma''
    \]
    und
    \[
        k_1 + k_2 = k
    \]
\end{lemma}
\begin{proof}
    Induktion nach $k$.

    \emph{Induktionsanfang:}
    \begin{enumerate}
        \item $k = 1$: Voraussetzung kann dafür nicht erfüllt sein, also stimmt die Aussage.
        \item $k = 2$: Wie kann $\langle S_1; S_2, \sigma \rangle \Rightarrow^2 \sigma''$ gelten?
            Das kann nur sein, wenn im ersten Schritt $\infruleSos[2]{seq}$ augewendet wird. D.\,h. die Schlussregel
            \[
                \frac{\stranssos{S_1}{\sigma}{\sigma'}}{\stranssos{S_1; S_2}{\sigma}{\langle S_2, \sigma' \rangle}}
            \]
            wurde erfüllt für ein $\sigma'$.

            Wir wissen also, es existiert ein Zwischenzustand $\sigma'$ mit $\stranssos{S_1}{\sigma}{\sigma'}$ und erster Schritt von $\langle S_1; S_2, \sigma \rangle \Rightarrow^2 \sigma''$ ist $\langle S_1; S_2, \sigma \rangle \Rightarrow \langle S_2, \sigma' \rangle$
            Der zweite Schritt $\langle S_1; S_2, \sigma \rangle \Rightarrow^2 \sigma''$ muss jetzt aber der Form $\langle S_2, \sigma' \rangle \Rightarrow \sigma''$ sein.

            D.\,h. die Aussage gilt mit $k_1 = 1, k_2 = 1$ und $\sigma'$.
    \end{enumerate}

        \par\bigskip
    \emph{Induktionsschritt:} $k - 1 \mapsto k$ mit $k \geq 3$

    Betrachten den ersten Schritt $\langle S_1; S_2, \sigma \rangle \Rightarrow^k \sigma''$.
    Zwei Fälle
    \begin{enumerate}
        \item $\infruleSos[1]{seq}$ Der erste Schritt hat die Form $\langle S_1; S_2, \sigma \rangle \Rightarrow^k \langle S_1'; S_2, \sigma''' \rangle$

            Dann muss aber gelten $\langle S_1'; S_2, \sigma''' \rangle \Rightarrow^{k-1} \sigma''$.
            Nach IV existiert $k_1', k_2' \in \mathbb{N}, \sigma'$, sodass $\langle S_1',  \sigma''' \rangle \Rightarrow^{k_1'} \sigma'$ und $\langle S_2',  \sigma' \rangle \Rightarrow^{k_2'} \sigma''$ und $k_1' + k_2' = k - 1$.

            Da wir im ersten Schritt $\infruleSos[1]{seq}$ angewandt haben, muss die Schlussregel dafür erfüllt gewesen sein, \dh{} es gilt $\langle S_1, \sigma \rangle \Rightarrow \langle S_1', \sigma''' \rangle$. Also gilt auch $\langle S_1', \sigma \rangle \Rightarrow^{k_1'} \sigma'$ also $\langle S_1, \sigma \rangle \Rightarrow^{k_1+1} \sigma'$.

            Also gilt die Aussage für $k_1 + 1, k_2 = k_2', \sigma'$.
        \item $\infruleSos[2]{seq}$ Der erste Schritt hat die Form $\langle S_1; S_2, \sigma \rangle \Rightarrow \langle S_2, \sigma' \rangle$ und es gilt $\langle S_1, \sigma \rangle \Rightarrow \sigma'$.

            Also gilt die Aussage für $k_1 = 1, k_2 = k - 1, \sigma'$.
    \end{enumerate}
\end{proof}


\begin{lemma}[Determinierheit]
    SOS ist \emph{determiniert}. Anders als bei der natürlichen Semantik müssen auch alle Zwischenzustände gleich sein, \dh{}

        Für jedes $S, \sigma$ existiert gibt es eine Ableitungsfolge, die mit $\langle S, \sigma \rangle$ beginnt.
\end{lemma}


\begin{definition}[Semantische Äquivalenz]
    Seien $S_1, S_2$ zwei Anweisungen. $S_1, S_2$ heißen \emph{semantische äquivalent} gdw. folgendes für allen Zustände $\sigma$ gilt:

    \begin{enumerate}
        \item Für alle steckengebliebenen Konfigurationen $\gamma$ und alle Endzustände $\sigma'$ gilt
            \[
                \langle S_1, \sigma \rangle \Rightarrow^* \gamma \Leftrightarrow \langle S_2, \sigma \rangle \Rightarrow^* \gamma
            \]
            und
            \[
                \langle S_1, \sigma \rangle \Rightarrow^* \sigma' \Leftrightarrow \langle S_2, \sigma \rangle \Rightarrow^* \sigma'
            \]
        \item Es existiert eine undendliche Ableitungsfolge für $\langle S_1, \sigma \rangle$ gdw. es existiert eine unendliche Folge für $\langle S_2, \sigma \rangle$.
    \end{enumerate}
\end{definition}

\par\medskip
\begin{example}
    \texttt{$S_1$; ($S_2$; $S_3$)} und \texttt{($S_1$; $S_2$); $S_3$} sind semantische äquivalent.
\end{example}



\subsection{Semantische Funktion $\mathcal{S_{\text{sos}}}$}

\begin{definition}
    Definiere $\mathcal{S_{\text{sos}}}: \SExp \to (\State \to \State)$ als
    \[
        \mathcal{S_{\text{sos}}}\lsem S \rsem(\sigma) = \begin{cases}
            \sigma' & \text{falls} \langle S, \sigma \rangle \Rightarrow^* \sigma' \\
            \bot & \text{sonst}
        \end{cases}
    \]

    Diese Funktion ist wohldefiniert, da SOS determiniert ist.
\end{definition}

\begin{theorem}
    Sei $S$ eine Anweisung und seien $\sigma, \sigma'$ Zustände. Dann gilt
    \[
        \mathcal{S_{\text{ns}}}\lsem S \rsem(\sigma) = \sigma'
        \quad\Leftrightarrow\quad
        \mathcal{S_{\text{sos}}}\lsem S \rsem(\sigma) = \sigma'
    \]

    D.\,h. SOS und NS sind äquivalent für unser \emph{konkretes Beispiel} der \texttt{while}-Sprache.
\end{theorem}

\begin{proof}
    Zwei Richtungen:
    \begin{enumerate}
        \item ``$\Rightarrow$'': $\mathcal{S_{\text{ns}}}\lsem S \rsem(\sigma) = \sigma' \Rightarrow \mathcal{S_{\text{sos}}}\lsem S \rsem(\sigma) = \sigma'$

            \dh{} $\langle S, \sigma \rangle \to \sigma' \implies \langle S, \sigma \rangle \Rightarrow^* \sigma'$

            Wir wissen $\langle S, \sigma \rangle \to \sigma'$, \dh{} es existiert ein endliche Ableitungsbaum $T$ dafür. Mache Induktion nach der Tiefe von $T$.

            \par\medskip
            \emph{Induktionsanfang:} $T$ hat Tiefe 0, \dh{} $T$ besteht nur aus einer Wurzel. Das bedeutet, $\langle S, \sigma \rangle \to \sigma'$ erfolgt durch Anwendung eines Axioms. Davon gibt es drei Stück: $\infruleNs{zuw}, \infruleNs{skip}, \infruleNs[\false]{while}$

            Exemplarisch für $\infruleNs[\false]{while}$:

            Wir wissen $S$ hat die Form \texttt{while $b$ do $S$} und $\Bsem{b}(\sigma) = \false$. D.\,h. $T$ hat die Form
            \[
                \frac{}{
                    \strans{\texttt{while $b$ do $S'$}}{\sigma}{\underbrace{\sigma'}_{\sigma}}
                }
            \]
            Jetzt gilt
            \begin{align*}
                & \langle \texttt{while $b$ do $S'$}, \sigma \rangle \\
                \overset{\infruleSos[]{while}}{\Rightarrow} \quad & \langle \texttt{if $b$ then ($S'$; while $b$ do $S'$) else skip} , \sigma \rangle \\
                \overset{\infruleSos[\false]{if}}{\Rightarrow} \quad & \langle \texttt{skip}, \sigma \rangle \quad\quad \text{da } \Bsem{b}(\sigma) = \false \\
                \overset{\infruleSos{skip}}{\Rightarrow} \quad & \sigma \\
            \end{align*}
            Also $\langle \texttt{while $b$ do $S'$}, \sigma \rangle \Rightarrow^* \sigma$ wie gewünscht.

            \par\bigskip
            \emph{Induktionsschritt:}

    \end{enumerate}
\end{proof}