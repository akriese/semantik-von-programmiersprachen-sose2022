\subsection{Strukturelle operationelle Semantik (``small step'')}

Hier geht es um die genaue Reihenfolge der Schritte bei der Ausführung. Das ist beispielsweise nützlich bei der parallelen Ausführungen eines Programms.

Wir definieren wieder eine Zustandsüberführungsrelation ``$\Rightarrow$''. Sie hat die Form
\begin{align*}
    \langle S, \sigma \rangle & \Rightarrow \langle S', \sigma' \rangle \tag{*} \\
    \text{oder} \\
    \langle S, \sigma \rangle & \Rightarrow \sigma' \tag{**}
\end{align*}

Interpretation:
\begin{enumerate}
    \item[(*)] Ausführung ist noch nicht vorbei, sondern erreicht in \emph{einem Schritt} die \emph{Zwischenkonfiguration} $\langle S', \sigma' \rangle$.
    \item[(**)] Ausführung ist nach einem Schritt vorbei und erreicht den Endzustand $\sigma'$.
\end{enumerate}



Wir definieren $\Rightarrow$ durch folgende Schlussregeln. Im Folgenden schreiben wir $[\cdot]_{\text{sos}}$ um anzuzeigen, dass es sich um die strukturelle operationelle Semantik handelt.


\begin{enumerate}
    \item $\infruleSos{zuw}$
    \[
    \langle x \texttt{ := } a, \sigma \rangle \Rightarrow \sigma[x \mapsto \Asem{a}(\sigma)]
    \]

    \item $\infruleSos{skip}$
    \[
    \langle \texttt{skip}, \sigma \rangle \Rightarrow \sigma
    \]

    \item $\infruleSos[1]{seq}$
    \begin{align*}
        \frac{
            \langle S_1, \sigma \rangle \Rightarrow \langle S_1', \sigma' \rangle
        }{
            \langle S_1\texttt{;} S_2, \sigma \rangle \Rightarrow \langle S_1'\texttt{;} S_2, \sigma' \rangle
        }
    \end{align*}

    \item $\infruleSos[2]{seq}$
    \begin{align*}
        \frac{
            \langle S_1, \sigma \rangle \Rightarrow \sigma'
        }{
            \langle S_1\texttt{;} S_2, \sigma \rangle \Rightarrow \langle S_2, \sigma' \rangle
        }
    \end{align*}


    \item $\infruleSos[\true]{if}$
    \begin{align*}
        \langle \texttt{if } b \texttt{ then } S_1 \texttt{ else } S_2, \sigma \rangle \Rightarrow \langle S_1, \sigma \rangle
    \end{align*}
    falls $\Bsem{b}(\sigma) = \true$

    \item $\infruleSos[\false]{if}$
    \begin{align*}
        \langle \texttt{if } b \texttt{ then } S_1 \texttt{ else } S_2, \sigma \rangle \Rightarrow \langle S_2, \sigma \rangle
    \end{align*}
    falls $\Bsem{b}(\sigma) = \false$

    \item $\infruleSos{while}$
    \[
    \langle \texttt{while } b \texttt{ do } S, \sigma \rangle \Rightarrow \langle \texttt{if } b \texttt{ then } \texttt{(S; while } b \texttt{ do } S) \texttt{ else skip}, \sigma \rangle
    \]
\end{enumerate}

\begin{definition}
    Sei $S$ eine Anweisung und $\sigma$ ein Zustand.

    Eine Ableitungsfolge für $\langle S, \sigma \rangle$ ist entweder
    \begin{enumerate}
        \item eine endliche Folge $\gamma_0 \Rightarrow \gamma_1 \Rightarrow \dots \Rightarrow \gamma_k$ von Konfigurationen, sodass $\gamma_0 = \langle S, \sigma \rangle$ ist, $\gamma_i \Rightarrow \gamma_{i+1}$ für $0, \dots, k-1$ gilt und $\gamma_k$ entweder ein Zustand $\sigma'$ oder eine Konfiguration $\langle S', \sigma' \rangle$ ist, für die es mit $\Rightarrow$ nicht weiter geht \emph{(steckengebliebene Konfiguration)}.

        \item eine unendliche Folge $\gamma_0 \Rightarrow \gamma_1 \Rightarrow \dots$ von Konfigurationen mit $\gamma_0 = \langle S, \sigma \rangle$ mit $\gamma_i \Rightarrow \gamma_{i+1}$ für $i \geq 0$.
    \end{enumerate}
\end{definition}

\par\medskip
\begin{notation}
    Wir schreiben

    $\gamma_0 \Rightarrow^i \gamma'$ für ``$\gamma'$ geht aus $i$ Schritten hervor'' und

    $\gamma_0 \Rightarrow^* \gamma'$ für ``$\gamma'$ geht aus endlich vielen Schritten hervor'' (auch null).
\end{notation}



%%%%%%%%%%%%%%%%%%%%%%%%%%%%%%%%%%%%%%%%%%%%%%%%%%%%%%%%%%%%%%%%%%%%%
\newpage
\hfill 03.06.
%%%%%%%%%%%%%%%%%%%%%%%%%%%%%%%%%%%%%%%%%%%%%%%%%%%%%%%%%%%%%%%%%%%%%

\begin{example}
    Sei $\sigma$ ein Zustand mit $\sigma(x) = 5, \sigma(y) = 7$. Betrachte die Auswertung von \texttt{(z := x; x := y); y = z;} in der SOS für Startzustand $\sigma$.

    \begin{align*}
        & \langle \texttt{(z := x; x := y); y = z;}, \sigma \rangle \\
        \overset{\infruleSos[1]{seq}\infruleSos[2]{seq}\infruleSos{zuw}}{\Rightarrow} \quad & \langle \texttt{x := y; y := z}, \sigma[z \mapsto 5] \rangle \tag{i} \\
        \overset{\infruleSos[2]{seq}\infruleSos{zuw}}{\Rightarrow} \quad & \langle \texttt{y := z}, \sigma[z \mapsto 5][y \mapsto 7] \rangle \tag{ii} \\
        \overset{\infruleSos{zuw}}{\Rightarrow} \quad & \sigma[z \mapsto 5][x \mapsto 7][y \mapsto 5]
    \end{align*}

    zu (i):
    \begin{align*}
        \infruleSos[1]{seq}\; \cfrac{
            \infruleSos[2]{seq}\; \cfrac{
                \infruleSos{zuw}\; \cfrac{}{
                    \stranssos{\texttt{z := x}}{\sigma}{\sigma[z \mapsto 5]}
                }
            }{
                \stranssos{\texttt{z := x; x := y}}{\sigma}{\langle \texttt{x := y}, \sigma[z \mapsto 5] \rangle}
            }
        }{
            \stranssos{\texttt{(z := x; x := y); y := z}}{\sigma}{\langle \texttt{x := y; y := z}, \sigma[z \mapsto 5] \rangle}
        }
    \end{align*}

    zu (ii):
    \begin{align*}
        \infruleSos[2]{seq}\; \cfrac{
            \infruleSos{zuw}\; \cfrac{}{
                \stranssos{x :=y}{\sigma[z \mapsto 5]}{\langle y :=z, \sigma[z \mapsto5][x \mapsto 7] \rangle}
            }
        }{
            \stranssos{x :=y; y :=z}{\sigma[z \mapsto 5]}{\langle y :=z, \sigma[z \mapsto5][x \mapsto 7] \rangle}
        }
    \end{align*}
\end{example}



\subsection{Eigenschaften der SOS}

\begin{lemma}
    Seien $S_1, S_2$ Anweisungen, $\sigma, \sigma''$ Zustände und $k \in \mathbb{N}$.
    Dann gilt: Falls $\langle S_1; S_2, \sigma \rangle \Rightarrow^k \sigma''$ gilt, es existieren zwei Zahlen $k_1, k_2 \in \mathbb{N}$ mit
    \[
        \langle S_1, \sigma \rangle \Rightarrow^{k_1} \sigma'
        \;,\quad
        \langle S_2, \sigma' \rangle \Rightarrow^{k_2} \sigma''
    \]
    und
    \[
        k_1 + k_2 = k
    \]
\end{lemma}
\begin{proof}
    Induktion nach $k$.

    \emph{Induktionsanfang:}
    \begin{enumerate}
        \item $k = 1$: Voraussetzung kann dafür nicht erfüllt sein, also stimmt die Aussage.
        \item $k = 2$: Wie kann $\langle S_1; S_2, \sigma \rangle \Rightarrow^2 \sigma''$ gelten?
            Das kann nur sein, wenn im ersten Schritt $\infruleSos[2]{seq}$ augewendet wird. D.\,h. die Schlussregel
            \[
                \frac{\stranssos{S_1}{\sigma}{\sigma'}}{\stranssos{S_1; S_2}{\sigma}{\langle S_2, \sigma' \rangle}}
            \]
            wurde erfüllt für ein $\sigma'$.

            Wir wissen also, es existiert ein Zwischenzustand $\sigma'$ mit $\stranssos{S_1}{\sigma}{\sigma'}$ und erster Schritt von $\langle S_1; S_2, \sigma \rangle \Rightarrow^2 \sigma''$ ist $\langle S_1; S_2, \sigma \rangle \Rightarrow \langle S_2, \sigma' \rangle$.
            Der zweite Schritt $\langle S_1; S_2, \sigma \rangle \Rightarrow^2 \sigma''$ muss jetzt aber der Form $\langle S_2, \sigma' \rangle \Rightarrow \sigma''$ sein.

            D.\,h. die Aussage gilt mit $k_1 = 1, k_2 = 1$ und $\sigma'$.
    \end{enumerate}

        \par\bigskip
    \emph{Induktionsschritt:} $k - 1 \mapsto k$ mit $k \geq 3$

    Betrachten den ersten Schritt $\langle S_1; S_2, \sigma \rangle \Rightarrow^k \sigma''$.
    Zwei Fälle:
    \begin{enumerate}
        \item $\infruleSos[1]{seq}$ Der erste Schritt hat die Form $\langle S_1; S_2, \sigma \rangle \Rightarrow \langle S_1'; S_2, \sigma''' \rangle$

            Dann muss aber gelten $\langle S_1'; S_2, \sigma''' \rangle \Rightarrow^{k-1} \sigma''$.
            Nach IV existiert $k_1', k_2' \in \mathbb{N}, \sigma'$, sodass $\langle S_1',  \sigma''' \rangle \Rightarrow^{k_1'} \sigma'$ und $\langle S_2',  \sigma' \rangle \Rightarrow^{k_2'} \sigma''$ und $k_1' + k_2' = k - 1$.

            Da wir im ersten Schritt $\infruleSos[1]{seq}$ angewandt haben, muss die Schlussregel dafür erfüllt gewesen sein, \dh{} es gilt $\langle S_1, \sigma \rangle \Rightarrow \langle S_1', \sigma''' \rangle$. Also gilt auch $\langle S_1', \sigma \rangle \Rightarrow^{k_1'} \sigma'$ und somit $\langle S_1, \sigma \rangle \Rightarrow^{k_1'+1} \sigma'$.

            Also gilt die Aussage für $k_1 = k_1' + 1, k_2 = k_2', \sigma'$.
        \item $\infruleSos[2]{seq}$ Der erste Schritt hat die Form $\langle S_1; S_2, \sigma \rangle \Rightarrow \langle S_2, \sigma' \rangle$ und es gilt $\langle S_1, \sigma \rangle \Rightarrow \sigma'$.

            Also gilt die Aussage für $k_1 = 1, k_2 = k - 1, \sigma'$.
    \end{enumerate}

    % https://github.com/jneuendorf/semantik-von-programmiersprachen-sose2022/pull/2#discussion_r931513559
    Wir unterscheiden also zwischen den beiden Möglichkeiten für die Sequenz. Im ersten Fall benötigen wir mehr als einen Schritt, um $S_1$ abzuarbeiten, im zweiten Fall benötigt $S_1$ genau einen Schritt.
\end{proof}


\begin{lemma}[Determiniertheit]
    SOS ist \emph{determiniert}. Anders als bei der natürlichen Semantik müssen auch alle Zwischenzustände gleich sein, \dh{}:
\par\medskip
\begin{lemma}[Determiniertheit]
    SOS ist \emph{determiniert}. Anders als bei der natürli"-chen Semantik müssen auch alle Zwischenzustände gleich sein, \dh{}:

        Für jedes $S, \sigma$ existiert eine Ableitungsfolge, die mit $\langle S, \sigma \rangle$ beginnt.
\end{lemma}


\par\medskip
\begin{definition}[Semantische Äquivalenz]
    Seien $S_1, S_2$ zwei Anweisungen. $S_1, S_2$ heißen \emph{semantisch äquivalent} gdw. folgendes für alle Zustände $\sigma$ gilt:

    \begin{enumerate}
        \item Für alle steckengebliebenen Konfigurationen $\gamma$ und alle Endzustände $\sigma'$ gilt
            \[
                \langle S_1, \sigma \rangle \Rightarrow^* \gamma \Leftrightarrow \langle S_2, \sigma \rangle \Rightarrow^* \gamma
            \]
            und
            \[
                \langle S_1, \sigma \rangle \Rightarrow^* \sigma' \Leftrightarrow \langle S_2, \sigma \rangle \Rightarrow^* \sigma'
            \]
        \item Es existiert eine undendliche Ableitungsfolge für $\langle S_1, \sigma \rangle$ gdw. es existiert eine unendliche Folge für $\langle S_2, \sigma \rangle$.
    \end{enumerate}
\end{definition}

\par\medskip
\begin{example}
    \texttt{$S_1$; ($S_2$; $S_3$)} und \texttt{($S_1$; $S_2$); $S_3$} sind semantisch äquivalent.
\end{example}



\subsection{Semantische Funktion $\mathcal{S_{\text{sos}}}$}

\begin{definition}
    Definiere $\mathcal{S_{\text{sos}}}: \Stm \to (\State \to \State)$ als
    \[
        \mathcal{S_{\text{sos}}}\lsem S \rsem(\sigma) = \begin{cases}
            \sigma' & \text{falls } \langle S, \sigma \rangle \Rightarrow^* \sigma' \\
            \bot & \text{sonst}
        \end{cases}
    \]

    Diese Funktion ist wohldefiniert, da SOS determiniert ist.
\end{definition}

\begin{theorem}
    Sei $S$ eine Anweisung und seien $\sigma, \sigma'$ Zustände. Dann gilt
    \[
        \mathcal{S_{\text{ns}}}\lsem S \rsem(\sigma) = \sigma'
        \quad\Leftrightarrow\quad
        \mathcal{S_{\text{sos}}}\lsem S \rsem(\sigma) = \sigma'
    \]

    D.\,h. SOS und NS sind äquivalent für unser \emph{konkretes Beispiel} der \texttt{while}-Sprache.
\end{theorem}

\begin{proof}
    Zwei Richtungen:
    \begin{enumerate}
        \item ``$\Rightarrow$'': $\mathcal{S_{\text{ns}}}\lsem S \rsem(\sigma) = \sigma' \Rightarrow \mathcal{S_{\text{sos}}}\lsem S \rsem(\sigma) = \sigma'$

            \dh{} $\langle S, \sigma \rangle \to \sigma' \implies \langle S, \sigma \rangle \Rightarrow^* \sigma'$

            Wir wissen $\langle S, \sigma \rangle \to \sigma'$, \dh{} es existiert ein endlicher Ableitungsbaum $T$ dafür. Führe Induktion nach der Tiefe von $T$ durch.

            \par\medskip
            \emph{Induktionsanfang:} $T$ hat Tiefe 0, \dh{} $T$ besteht nur aus einer Wurzel. Das bedeutet, $\langle S, \sigma \rangle \to \sigma'$ erfolgt durch Anwendung eines Axioms. Davon gibt es drei Stück: $\infruleNs{zuw}, \infruleNs{skip}, \infruleNs[\false]{while}$

            Exemplarisch für $\infruleNs[\false]{while}$:

            Wir wissen $S$ hat die Form \texttt{while $b$ do $S$} und $\Bsem{b}(\sigma) = \false$. D.\,h. $T$ hat die Form
            \[
                \frac{}{
                    \strans{\texttt{while $b$ do $S'$}}{\sigma}{\underbrace{\sigma'}_{\sigma}}
                }
            \]
            Jetzt gilt
            \begin{align*}
                & \langle \texttt{while $b$ do $S'$}, \sigma \rangle \\
                \overset{\infruleSos{while}}{\Rightarrow} \quad & \langle \texttt{if $b$ then ($S'$; while $b$ do $S'$) else skip} , \sigma \rangle \\
                \overset{\infruleSos[\false]{if}}{\Rightarrow} \quad & \langle \texttt{skip}, \sigma \rangle \quad\quad \text{da } \Bsem{b}(\sigma) = \false \\
                \overset{\infruleSos{skip}}{\Rightarrow} \quad & \sigma \\
            \end{align*}
            Also $\langle \texttt{while $b$ do $S'$}, \sigma \rangle \Rightarrow^* \sigma$ wie gewünscht.

%%%%%%%%%%%%%%%%%%%%%%%%%%%%%%%%%%%%%%%%%%%%%%%%%%%%%%%%%%%%%%%%%%%%%
\newpage
\hfill 10.06.
%%%%%%%%%%%%%%%%%%%%%%%%%%%%%%%%%%%%%%%%%%%%%%%%%%%%%%%%%%%%%%%%%%%%%

            \par\bigskip
            \emph{Induktionsschritt:}
            \begin{enumerate}
                \item $\infruleNs[\true]{while}$
                \item $\infruleNs[*]{if}$
                \item $\infruleNs{seq}$
            \end{enumerate}

            Wir machen exemplarisch (a).

            Es gilt $\strans{S}{\sigma}{\sigma'}$, \dh{} $T$: \[
                \frac{
                    \cfrac{T_1}{\strans{S'}{\sigma}{\sigma''}}
                    \;,\quad
                    \cfrac{T_2}{\strans{\texttt{while b do S'}}{\sigma''}{\sigma'}}
                }{
                    \strans{\texttt{while b do S'}}{\sigma}{\sigma'}
                }
            \]
            $T_1$ und $T_2$ existieren, da $T$ existiert. Außerdem sind die Höhen von $T_1, T_2 < $ Höhe von $T$. Also folgt aus der IV, dass

            wegen $\strans{S'}{\sigma}{\sigma''}$ auch \[
                \langle S', \sigma \rangle \Rightarrow^* \sigma''
            \]
            und wegen $\strans{S'}{\sigma''}{\sigma'}$ auch \[
                \langle S, \sigma'' \rangle \Rightarrow^* \sigma'
            \]
            gilt.

            \begin{align*}
                \langle \texttt{while b do S'}, \sigma \rangle \Rightarrow & \quad \langle \texttt{if b then (S'; while b do S') else skip}, \sigma \rangle \\
                \overset{\infruleSos[\true]{if}}{\Rightarrow} & \quad \langle \texttt{S'; while b do S'}, \sigma \rangle \quad \text{weil } \Bsem{b}(\sigma) = \true \\
                \overset{\text{IV}}{\Rightarrow^*} & \quad \sigma' \quad \text{(Übung)}
            \end{align*}

        \item ``$\Leftarrow$'': Übung
    \end{enumerate}
\end{proof}



\subsection{Erweiterungen der \texttt{while}-Sprache}


\subsubsection{Programmabbruch}

Füge eine neue Anweisung \texttt{abort} zur Sprache hinzu. Die Bedeutung ist, dass das Programm fehlerhaft abgebrochen wird.
\[
    S \to \dots \;\vert\; \texttt{abort}
\]
Wir fügen für \texttt{abort} \emph{keine neuen Schlussregeln} hinzu -- weder für die NS noch die SOS).

\begin{example}
    Wir betrachten folgende Anweisung, in der die Schleife die Fakultät berechnet:

    $S =$ \texttt{if $\neg(x \geq 1)$ then abort else (y := 1; while $\neg(x = 1)$ do (y := y * x; x := x - 1))}

    \par\bigskip
    \hrule
    \emph{Wie sieht es bei der NS aus?}

    Sie $\sigma(x) = -1$. Wie sieht dann der Ableitungsbaum aus?
    \begin{align*}
        \infruleNs[\true]{if}\; \frac{
            \cfrac{
                T
            }{
                \strans{\texttt{abort}}{\sigma}{\sigma'}
            }
        }{
            \strans{S}{\sigma}{\sigma'}
        }
    \end{align*}
    Dabei existiert $T$ nicht! Daraus folgt \[
        \Ssem{S}(\sigma) = \bot
    \] (für unseren konkreten Zustand $\sigma$).

    $S$ ist \textbf{semantisch äquivalent} zu

    $S' =$ \texttt{if $\neg(x \geq 1)$ then (while true do skip) else (y := 1; while $\neg(x = 1)$ do (y := y * x; x := x - 1))}

    denn auch für \texttt{while true do skip} existiert auch kein endlicher Ableitungsbaum. D.\,h. in der natürlichen Semantik können wir nicht zwischen Abbruch und Endlossschleife unterscheiden.

    \par\bigskip
    \hrule
    \emph{Wie sieht es bei der SOS aus?}
    \begin{align*}
        \langle S, \sigma \rangle \overset{\infruleSos[\true]{if}}{\Rightarrow} \langle \texttt{abort}, \sigma \rangle
    \end{align*}
    Die rechte Konfiguration ist eine steckengebliebene Konfiguration.
    \begin{align*}
        \langle S', \sigma \rangle \overset{\infruleSos[\true]{if}}{\Rightarrow} & \quad \langle \texttt{while true do skip}, \sigma \rangle \\
        \Rightarrow^* & \quad \langle \texttt{while true do skip}, \sigma \rangle \\
        \Rightarrow^* & \quad \dots
    \end{align*}
    D.\,h. in der SOS sind $S$ und $S'$ \textbf{nicht semantisch äquivalent}.
\end{example}



\subsubsection{Blöcke und lokale Variablen}

\emph{Idee:} Erlaube Definition lokaler Variablen innerhalb eines Bereichs.

\begin{align*}
    S & \to \dots \;\vert\; \texttt{begin $D_V$ $S$ end} \\
    D_V & \to \texttt{var} \text{ Var } \texttt{:= A; $D_V$} \;\vert\; \varepsilon
\end{align*}

$D_V$ entspricht der syntaktischen Kategorie Dec$_V$.

\begin{example}
Betrachte
\begin{lstlisting}[language=Pascal]
begin
    var y := 1;
    x := 1;
    begin
        var x := 2;
        y := x + 1  (* (1) 2 + 1 *)
    end
    y := y + x (* (2) 3 + 1 *)
end
\end{lstlisting}

In (1) überdeckt das lokale $x$ das globale $x$ \emph{in diesem Block}. In (2) wird wieder das globale $x$ verwendet.
\end{example}



\subsubsection{Formalisierung lokaler Blöcke in der NS}

\begin{definition}
    Sei $D_V$ eine Variablendeklaration. Die Menge der in ihr deklarierten Variablen $\DV(D_V)$ ist definiert als:
    \begin{align*}
        \DV(\varepsilon) & := \varnothing \\
        \DV(\texttt{var } x := a \texttt{; } D_V') & := \{ x \} \cup \DV(D_V')
    \end{align*}

    Wir führen eine neue Relation ``$\to_D$'' ein. Sie hat die Form \[
        \langle D_V, \sigma \rangle \to_D \sigma'
    \]
    wobei $\sigma'$ ein lokaler Zustand ist.
\end{definition}



\subsubsection{Schlussregeln für $D_V$, $\to_D$ und Blöcke}

\begin{align*}
    \frac{}{\langle \varepsilon, \sigma \rangle \to_D \sigma} \\
    \\
    \frac{
        \langle D_V, \sigma[x \mapsto \Asem{a}(\sigma)] \rangle \to_D \sigma'
    }{
        \langle \texttt{var } x := a \texttt{; } D_V, \sigma \rangle \to_D \sigma'
    } \\
    \\
    \frac{
        \langle D_V, \sigma \rangle \to_D \sigma'', \langle S, \sigma'' \rangle \to \sigma'
    }{
        \langle \texttt{begin $D_V$ $S$ end}, \sigma \rangle \to \sigma'[\DV(D_V) \mapsto \sigma]
    }
\end{align*}
wobei
\begin{align*}
    \sigma'[X \mapsto \sigma](x) = \begin{cases}
        \sigma(x) & \text{falls } x \in X \\
        \sigma'(x) & \text{falls } x \not\in X
    \end{cases}
\end{align*}
\dh{} alle lokal deklarierten Variablen werden auf ihren ursprünglichen Wert zurückgesetzt.


%%%%%%%%%%%%%%%%%%%%%%%%%%%%%%%%%%%%%%%%%%%%%%%%%%%%%%%%%%%%%%%%%%%%%
\newpage
\hfill 17.06.
%%%%%%%%%%%%%%%%%%%%%%%%%%%%%%%%%%%%%%%%%%%%%%%%%%%%%%%%%%%%%%%%%%%%%

\subsubsection{Unterprogramme / Prozeduren / Subroutine / Methode}

Wir wollen ein neues Sprachkonstrukt für \texttt{while} einführen: Prozeduren (wie in Pascal).

Dazu erweitern wir im ersten Schritt die Grammatik:
\begin{align*}
    S & \to \dots \;\vert\; \texttt{begin $D_V$ $D_p$ $S$ end} \;\vert\; \texttt{call $p$} \\
    D_p & \to \texttt{proc $p$ is $S$; $D_p$} \;\vert\; \varepsilon
\end{align*}

$p$ steht für den Bezeichner der Prozedur. $D_p$ entspricht der syntaktischen Kategorie $p_{\text{Name}}$.
$D_p$ entspricht der syntaktischen Kategorie für Unterprogramm-Deklarationen.

\par\medskip
\begin{example}
Betrachte
\begin{lstlisting}[language=Pascal, caption=Programm mit Unterprogrammen]
begin
    var x := 0;
    proc p is x := x * 2;
    proc q is call p;

    begin
        var x := 5;
        proc p is x := x + 1;
        call q;
        y := x
    end
end
\end{lstlisting}

\emph{Frage: Was soll hier herauskommen?}

Das hängt von den Regeln über die Gültigkeitsbereiche ab (\emph{scope rules}).

Beim Aufruf von $q$ gibt es zwei Möglichkeiten: statisches Scoping (Bindung durch Syntax) oder dynamisches Scoping (Bindung durch aktuellen Kontext bei der Ausführung).

Wir müssen uns also entscheiden: Wenn ein Bezeichner in einer Funktion verwendet wird, bezieht er sich auf die Umgebung im Quelltext (\emph{statischer Gültigkeitsbereich}) oder auf die Umgebung zum Zeitpunkt der Programmausführung (\emph{dynamischer Gültigkeitsbereich})?

Diese Entscheidung kann man separat für jede Klasse von Bezeichnern treffen.
\end{example}

\begin{table}[H]
    \centering
    \begin{tabular}{c||c|c}
        Unterprogramme \textbackslash Variablen & statisch & dynamisch \\ \hline \hline
        statisch  & $y = 5$ & $y = 10$ \\ \hline
        dynamisch & $y = 6$ & $y = 6$
    \end{tabular}
    \caption{Ergebnisse in Abhängigkeit von den scoping rules}
\end{table}



\subsubsection{Natürlichen operationelle Semantik für Unterprogramme mit dynamischem Gültigkeitsbereich}

Um die Zuordnung von Prozedurname und Prozedurkörper (\emph{procedure body}) zu verwalten, müssen wir die Übergangsrelation ``$\to$'' um einen \emph{lokalen Kontext} erweitern. Dieser ist als Funktion \[
\Env_p: p_{\text{Name}} \to \Stm
\] formalisiert. Die Übergangsrelation hat nun die Form \[
    \env_p \vdash \langle S, \sigma \rangle \to \sigma'
\] mit $\env_p \in \Env_p$ (also eine Funktion), $S \in \Stm$ und $\sigma, \sigma' \in \State$. Das neue Symbol ``$\vdash$'' bedeutet ``im Kontext von''.

\emph{Interpretation:} Im lokalen Kontext $\env_p$ und mit Startzustand $\sigma$ terminiert die Anweisung $S$ nach endliche vielen Schritten und erreicht den Endzustand $\sigma'$.

\par\medskip
\begin{remark}
    Für statische Gültigkeitsbereiche müssten wir auch für Variablen einen Kontext erzeugen.
\end{remark}




\subsubsection{Schlussregeln für Unterprogramme mit dynamischem Gültigkeitsbereich}

Die meisten Schlussregeln werden übernommen, nur dass $\env_p$ ``mitgeschleppt'' wird.

\begin{example}
    $\infruleNs{seq}$ \begin{align*}
        \frac{
        \env_p \vdash \langle S_1, \sigma \rangle \to \sigma', \env_p \vdash \langle S_2, \sigma' \rangle \to \sigma''
        }{
            \env_p \vdash \langle \texttt{$S_1$; $S_2$}, \sigma \rangle \to \sigma''
        }
    \end{align*}
\end{example}

Wichtig / interessant sind nun $\infruleNs{block}$ und $\infruleNs{call}$ (neu):
\begin{enumerate}
    \item $\infruleNs{block}$ \begin{align*}
        \frac{
            \langle D_V, \sigma \rangle \to_D \sigma'',
            \quad
            \akt_p(D_p, \env_p) \vdash \langle S, \sigma'' \rangle \to \sigma'
        }{
            \env_p \vdash \langle \texttt{begin $D_V$ $D_p$ $S$ end}, \sigma \rangle \to \sigma'[DV(D_V) \mapsto \sigma]
        }
    \end{align*}
    Dabei bedeutet $\akt_p$: ``Passe den lokalen Kontext an!'' und ist wie folgt definiert:
    \begin{align*}
        \akt_p: D_p \times \Env_p & \to \Env_p \\
        \akt_p(\varepsilon, \env_p) & = \env_p \\
        \akt_p(\texttt{proc $p$ is $S$; $D_p$}, \env_p) & = \akt_p(D_p, \env_p[p \mapsto S])
    \end{align*}

    \item $\infruleNs{call}$ \begin{align*}
        \frac{
            \env_p \vdash \langle S, \sigma \rangle \to \sigma'
        }{
            \env_p \vdash \langle \texttt{call $p$}, \sigma \rangle \to \sigma'
        }
    \end{align*}
    wobei $\env_p(p) = S$, \dh{} nur wenn $p$ im aktuellen Kontext deklariert ist. Ansonsten kann man diese Regel nicht anwenden, \dh{} die Anweisung würde dann steckenbleiben.
\end{enumerate}

\begin{remark}
    Bei $\to_D$ benötigen wir keinen lokalen Kontext.

    Alles ist dynamisch, \dh{} $\env_p$ und $\sigma$ hängen vom aktuellen Programmzustand ab.

    Mit mehr Arbeit ($\leadsto$ zusätzliche Indirektion und umfangreicherer lokaler Kontext) kann man auch statische Gültigkeitsbereiche modellieren.
\end{remark}
