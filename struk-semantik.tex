\subsection{Strukturelle operationelle Semantik (``small step'')}

Hier geht es um die genaue Reihenfolge der Schritte bei der Ausführung. Das ist bespielsweise nützlich bei der parallelen Ausführungen eines Programms.

Wir definieren wieder eine Zustandsüberführungsrelation ``$\Rightarrow$''. Sie hat die Form
\begin{align*}
    \langle S, \sigma \rangle & \Rightarrow \langle S', \sigma' \rangle \tag{*} \\
    \text{oder} \\
    \langle S, \sigma \rangle & \Rightarrow \sigma' \tag{**}
\end{align*}

Interpretation:
\begin{enumerate}
    \item[(*)] Ausführung ist noch nicht vorbei, sondern erreicht in \emph{einem Schritt} die \emph{Zwischenkonfiguration} $\langle s', \sigma' \rangle$.
    \item[(**)] Ausführung ist nach einem Schritt vorbei und erreicht den Endzustand $\sigma'$.
\end{enumerate}



Wir definieren $\Rightarrow$ durch folgende Schlussregeln. Im Folgenden schreiben wir $[\cdot]_{\text{sos}}$ um anzuzeigen, dass es sich um die strukturelle operationelle Semantik handelt.


\begin{enumerate}
    \item $\infruleSos{zuw}$
    \[
    \langle x \texttt{ := } a, \sigma \rangle \Rightarrow \sigma[x \mapsto \Asem{a}(\sigma)]
    \]

    \item $\infruleSos{skip}$
    \[
    \langle \texttt{skip}, \sigma \rangle \Rightarrow \sigma
    \]

    \item $\infruleSos[1]{seq}$
    \begin{align*}
        \frac{
            \langle S_1, \sigma \rangle \Rightarrow \langle S_1', \sigma' \rangle
        }{
            \langle S_1\texttt{;} S_2, \sigma \rangle \Rightarrow \langle S_1'\texttt{;} S_2, \sigma' \rangle
        }
    \end{align*}

    \item $\infruleSos[2]{seq}$
    \begin{align*}
        \frac{
            \langle S_1, \sigma \rangle \Rightarrow \sigma'
        }{
            \langle S_1\texttt{;} S_2, \sigma \rangle \Rightarrow \langle S_2, \sigma' \rangle
        }
    \end{align*}


    \item $\infruleSos[\true]{if}$
    \begin{align*}
        \langle \texttt{if } b \texttt{ then } S_1 \texttt{ else } S_2, \sigma \rangle \Rightarrow \langle S_1, \sigma \rangle
    \end{align*}
    falls $\Bsem{b}(\sigma) = \true$

    \item $\infruleSos[\false]{if}$
    \begin{align*}
        \langle \texttt{if } b \texttt{ then } S_1 \texttt{ else } S_2, \sigma \rangle \Rightarrow \langle S_2, \sigma \rangle
    \end{align*}
    falls $\Bsem{b}(\sigma) = \false$

    \item $\infruleSos{while}$
    \[
    \langle \texttt{while } b \texttt{ do } S, \sigma \rangle \Rightarrow \langle \texttt{if } b \texttt{ then } \texttt{(S; while } b \texttt{ do } S) \texttt{ else skip}, \sigma \rangle
    \]
\end{enumerate}

\begin{definition}
    Sei $S$ eine Anweisung und $\sigma$ ein Zustand.

    Eine Ableitungsfolge für $\langle S, \sigma \rangle$ ist entweder
    \begin{enumerate}
        \item eine endliche Folge $\gamma_0 \Rightarrow \gamma_1 \Rightarrow \dots \Rightarrow \gamma_k$ von Konfigurationen, sodass $\gamma_0 = \langle S, \sigma \rangle$ ist, $\gamma_i \Rightarrow \gamma_{i+1}$ für $0, \dots, k-1$ gilt und $\gamma_k$ entweder ein Zustand $\sigma'$ oder eine Konfiguration $\langle S', \sigma' \rangle$ ist, für die es mit $\Rightarrow$ nicht weiter geht \emph{(steckengebliebe Konfiguration)}.

        \item eine unendliche Folge $\gamma_0 \Rightarrow \gamma_1 \Rightarrow \dots$ von Konfigurationen mit $\gamma_0 = \langle S, \sigma \rangle$ mit $\gamma_i \Rightarrow \gamma_{i+1}$ für $i \geq 0$.
    \end{enumerate}
\end{definition}

\par\medskip
\begin{notation}
    Wir schreiben

    $\gamma_0 \Rightarrow^i \gamma'$ für ``$\gamma'$ geht aus $i$ Schritten hervor'' und

    $\gamma_0 \Rightarrow^* \gamma'$ für ``$\gamma'$ geht aus endlich vielen Schritten hervor'' (auch null).
\end{notation}
