%%%%%%%%%%%%%%%%%%%%%%%%%%%%%%%%%%%%%%%%%%%%%%%%%%%%%%%%%%%%%%%%%%%%%
\newpage
\hfill 29.04.
%%%%%%%%%%%%%%%%%%%%%%%%%%%%%%%%%%%%%%%%%%%%%%%%%%%%%%%%%%%%%%%%%%%%%

\section{Semantik} \label{sec:semantik}

Die \emph{Semantik} eines Programmes ist die Bedeutung eines Programmes einer Programmiersprache.

Ziel: Finde eine klare, einfache mathematische Methode, um einem Programm eine \emph{Bedeutung} zuzuordnen.

Motivation:
\begin{itemize}
    \item Verifikation:
        \begin{itemize}
            \item Erfüllt mein Programm die Spezifikation (tut es das, was es soll)?
            \item Setzt der Übersetzer/Interpretierer die Spezifikation der Sprache korrekt um?
        \end{itemize}
     \item Programmumformung
        \begin{itemize}
            \item Haben zwei unterschiedliche Programme die gleiche Bedeutung?
            \item Optimierung
        \end{itemize}
    \item Programmanalyse
        \begin{itemize}
            \item Ist das Programm ``sicher'' (secure vs. safe)?
            \item Ist das Programm ``effizient''?
        \end{itemize}
\end{itemize}

\begin{definition}[Programmierparadigma]
    Programmierparadigma: \zb deklarativ (``Was?'') (funktional vs. logisch), imperativ (``Wie?''). In verschiedenen Paradigmen haben (potenziell) Programme verschiedene Bedeutungen.
\end{definition}

Wir konzentrieren uns auf \emph{imperative} Programmierung.

\begin{question}
    Was ist die ``mathematische Bedeutung'' eines imperativen Programms?
\end{question}

\begin{question}[folgend]
    Was ist ein imperatives Programm?
\end{question}



\begin{lstlisting}[language=Python, caption=Imperatives Programm]
x = 1
y = x + 2
x = y + 5
for ...
\end{lstlisting}
% => Es gibt einen Zustand (alles, was im Speicher steht). Diesen ändert man mit Zuweisung


\begin{lstlisting}[language=Haskell, caption=Funktionales Programm]
foo :: Int -> Int
foo 0 = 1
foo x = x + 1
foo 3
\end{lstlisting}
% => kein Zustand, sondern es gibt einen Ausdruck, der ausgewertet wird

Das zentrale Konzept der imperativen Programmierung ist der \emph{Zustand} (state). Der Zustand ist der Inhalt aller Speicherzellen und Register, die Position des Programm"-zählers und der Zustand der Eingabe-/Ausgabe-Geräte.

Ein imperatives Programm ist eine Folge von \emph{Anweisungen} (statement / instruction). Diese haben \emph{Wirkungen} (effects), welche den Zustand verändern (selbst \texttt{nop} ändert den Programmzähler und somit den Zustand). Darüber hinasu gibt es Neben"-wir"-kungen bzw. Seiteneffekte (side effects). Es gibt unterschiedliche Arten von Anwei"-sungen:
\begin{itemize}
    \item Zuweisungen (direkte Änderung des Zustandes)
    \item Kontrollfluss (Änderung des Programmzählers: Verzweigungen, Schleifen, Funk"-tions"-aufrufe bzw. Sprünge)
    \item Eingabe / Ausgabe
\end{itemize}
