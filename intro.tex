%%%%%%%%%%%%%%%%%%%%%%%%%%%%%%%%%%%%%%%%%%%%%%%%%%%%%%%%%%%%%%%%%%%%%
\newpage
\hfill 22.04.
%%%%%%%%%%%%%%%%%%%%%%%%%%%%%%%%%%%%%%%%%%%%%%%%%%%%%%%%%%%%%%%%%%%%%

\section{Einführung}

Diese Vorlesung orientiert sich an folgender Literatur:
\begin{itemize}
    \item Semantics with applications, Hanne Riis Nielson and Flemming Nielson, 1999
    \item Semantik von Programmiersprachen, Elfriede Fehr, 1989
\end{itemize}



\subsection{Programmiersprachen}

Eine Programmiersprache ist eine künstliche, entworfene Sprache zur Kommunikation zwischen Mensch und Rechner. Davon gibt es heutzutage sehr viele.

Keine Programmiersprache ist perfekt: Es gibt verschiedene Ziele und Aspekte beim Entwurf von Programmiersprachen und somit verschiedene Vor- und Nachteile:
\begin{itemize}
    \item Komfort
    \begin{itemize}
        \item Ausdrucksfähigkeit vs. Nützlichkeit
        \item Lesbarkeit (\zb{} COBOL) vs. Prägnanz (\zb{} FORTRAN)
    \end{itemize}
    \item Vermeidung von Fehlern im Programm
    \item leichte algorithmische Verarbeitung ($\leadsto$ parsing)
    \item effizienter erzeugter Code
    \begin{itemize}
        \item Dies ist eigentlich Aufgabe des Übersetzers ($\leadsto$ Vorlesung Übersetzer-Bau)
        \item abstrakt vs. hardware-nah
    \end{itemize}
\end{itemize}

\par\bigskip
In dieser Vorlesung befassen wir uns mit der theoretische Analyse von Programmiersprachen. Konkret beinhaltet dies:
\begin{enumerate}
    \item einzelne Sprachkonstrukte mathematisch modellieren,
    \item Beziehung zwischen Kontrukten verstehen,
    \item Güte der Konstrukte beweisen,
    \item theoretische Garantien ableiten.
\end{enumerate}

\par\bigskip
Jede Programmiersprache besitzt drei wesentliche Eigenschaften:
\begin{itemize}
    \item Syntax
    \item Semantik
    \item Idiomatik
\end{itemize}



\subsubsection{Syntax}

Die Syntax ist eine endliche Folge von Zeichen über einem Alphabet $\Sigma$, \zb{} \texttt{x = 5 + 2;} aus dem Alphabet ASCII.

Dabei sind nicht alle Zeichenfolgen korrekte Programme.

Die \emph{konkrete Syntax} beschreibt Zeichenfolgen, die gültige Programme darstellen. Dafür werden eine kontextfreie Grammatik und zusätzliche Regeln (\zb{} dass nur zuvor deklarierte Variablen verwendet werden dürfen) benutzt.

Die \emph{abstrakte Syntax} ist die aus der Grammatik resultierende, hierarchische Struktur eines gültigen Programms. Diese wird durch einen Syntaxbaum dargestellt.
\begin{align*}
    \text{Block} - \text{Anweisung} - \text{Zuweisung} - \begin{cases}
        \text{Variable} - x \\
        \\
        \text{Ausdruck} - \oplus - \begin{cases}
            5 \\
            \\
            2
        \end{cases}
    \end{cases}
\end{align*}

Die \emph{Syntaxanalyse} erzeugt einen Syntaxbaum aus einem konkreten Programm. Dies ist Inhalt der Vorlesung Übersetzer-Bau. Hier, in dieser Vorlesung gehen wir davon aus, dass der abstrakte Syntaxbaum gegeben ist.



\subsubsection{Semantik}

Diese Eigenschaft beleuchten wir in dieser Vorlesung ab dem nächsten \secref{sec:semantik}.



\subsubsection{Idiomatik}

Die \emph{Idiomatik} umfasst Konventionen, Muster bzw. Faustregeln bei der Verwendung einer bestimmten Programmiersprache ($\leadsto$ pattern, anti-patterns, best practices). Somit macht sie in der Praxis den eigentlichen Gehalt\,/\,Kultur einer Programmiersprache aus.
